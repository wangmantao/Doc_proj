
\documentclass[utf8]{ctexart}
\usepackage{booktabs}
\usepackage{tikz}
\usepackage{geometry}
\usepackage{enumerate}
\usepackage{makecell}
\geometry{left=2.0cm,right=2.0cm,top=2.5cm,bottom=2.5cm}
\begin{document}

\begin{bfseries} 
 工作日志
\end{bfseries}

\section {日志记录方法}
每天的工作记录都以同样的格式打印出来,装入活页文件夹中


\section {ARM\_charge}
\subsection {FreeRTOS}
/home/wmt/opt/FreeRTOSv10.1.1/FreeRTOS \\
/home/wmt/opt/FreeRTOSv10.1.1/FreeRTOS/Demo/lwIP_MCF5235_GCC/lwip/contrib/port/FreeRTOS \\
/home/wmt/opt/FreeRTOSv10.1.1/FreeRTOS/Demo/lwIP_Demo_Rowley_ARM7/lwip-1.1.0/contrib/port/FreeRTOS \\
/home/wmt/opt/FreeRTOSv10.1.1/FreeRTOS/Demo/Common/ethernet/lwIP_130/contrib/port/FreeRTOS \\
/home/wmt/myProj/王满涛/stm32f103c8t6/rtos/FreeRTOSv10.1.1/FreeRTOS \\
/home/wmt/myProj/王满涛/stm32f103c8t6/rtos/FreeRTOSv10.1.1/FreeRTOS/Demo/lwIP_MCF5235_GCC/lwip/contrib/port/FreeRTOS \\
/home/wmt/myProj/王满涛/stm32f103c8t6/rtos/FreeRTOSv10.1.1/FreeRTOS/Demo/lwIP_Demo_Rowley_ARM7/lwip-1.1.0/contrib/port/FreeRTOS \\
/home/wmt/myProj/王满涛/stm32f103c8t6/rtos/FreeRTOSv10.1.1/FreeRTOS/Demo/Common/ethernet/lwIP_130/contrib/port/FreeRTOS \\

以上是完整引用的RTOS项目.

第一类是:将来本地所有的FreeRTOS项目都用这个源.

第二类是: Begine STM32 pdf书的示例,按指导在源文件项目,自行从freeRTOS官方项目下载的RTOS源.


/home/wmt/temp/STM32F4-FreeRTOS/FreeRTOS \\
/home/wmt/temp/test/STM32F4-FreeRTOS/FreeRTOS \\

这个内容是地github的一个项目:STM32 FreeRTOS; 它用到了FreeRTOS, 但用的很简介明了,是值得学习的.
但是我要用哪种方法学习freeRTOS呢?

1. <<Begine STM32>>

2. github stm32 freeRTOS

3. FreeRTOS的Demo


/home/wmt/STM32Cube/Repository/STM32Cube_FW_F1_V1.7.0/Middlewares/Third_Party/FreeRTOS \\
/home/wmt/STM32Cube/Repository/STM32Cube_FW_F1_V1.7.0/Projects/STM3210E_EVAL/Applications/FreeRTOS \\
/home/wmt/STM32Cube/Repository/STM32Cube_FW_F1_V1.7.0/Projects/STM3210C_EVAL/Applications/FreeRTOS  \\
/home/wmt/STM32Cube/Repository/STM32Cube_FW_F1_V1.7.0/Projects/STM32F103RB-Nucleo/Applications/FreeRTOS  \\
/home/wmt/STM32Cube/Repository/STM32Cube_FW_F4_V1.21.0/Middlewares/Third_Party/FreeRTOS \\
/home/wmt/STM32Cube/Repository/STM32Cube_FW_F4_V1.21.0/Projects/STM32F412G-Discovery/Applications/FreeRTOS \\
/home/wmt/STM32Cube/Repository/STM32Cube_FW_F4_V1.21.0/Projects/STM32F413ZH-Nucleo/Applications/FreeRTOS \\
/home/wmt/STM32Cube/Repository/STM32Cube_FW_F4_V1.21.0/Projects/STM32F429I-Discovery/Applications/FreeRTOS \\
/home/wmt/STM32Cube/Repository/STM32Cube_FW_F4_V1.21.0/Projects/STM32446E_EVAL/Applications/FreeRTOS \\
/home/wmt/STM32Cube/Repository/STM32Cube_FW_F4_V1.21.0/Projects/STM32F413H-Discovery/Applications/FreeRTOS \\
/home/wmt/STM32Cube/Repository/STM32Cube_FW_F4_V1.21.0/Projects/STM324x9I_EVAL/Applications/FreeRTOS \\
/home/wmt/STM32Cube/Repository/STM32Cube_FW_F4_V1.21.0/Projects/STM32469I-Discovery/Applications/FreeRTOS \\
/home/wmt/STM32Cube/Repository/STM32Cube_FW_F4_V1.21.0/Projects/STM32469I_EVAL/Applications/FreeRTOS \\
/home/wmt/STM32Cube/Repository/STM32Cube_FW_F4_V1.21.0/Projects/STM324xG_EVAL/Applications/FreeRTOS \\

以上是两个地方的cube库,上一个是最新的F1/F4的库,含有FreeRTOS的示例项目和第三方中间件

这个库是怎么安装进来的?安装它的软件,是否可以在本Linux系统下做IDE.或者要用到cube库时,从此处调用.





\end{document}
